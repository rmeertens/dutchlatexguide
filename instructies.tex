\documentclass[a4paper,10pt]{article}
\usepackage[utf8]{inputenc}
\usepackage{hyperref}
\usepackage{graphicx}
\usepackage{verbatimbox}
\usepackage{textcomp}

\title{\LaTeX Workshop}

\date{11 november, 17:45}

\author{Harmen Prins en Roland Meertens}

\begin{document}

\maketitle

Welkom bij deze workshop over \LaTeX. Ondanks dat deze workshop niet superuitgebreid is, hopen we dat je hier genoeg leert om je eigen documenten te maken en de andere dingen kan Googlen. Als je nog vragen of opmerkingen hebt, mag je altijd een mailtje sturen naar   \href{mailto:rolandmeertens@gmail.com}{Roland}. Succes!

\section{Share\LaTeX}
We gebruiken Share\LaTeX, dus maak maar een account:
\begin{itemize}
\item Ga naar \href{http://www.sharelatex.com}{sharelatex.com}
\item Vul je gegevens in en klik op register
\end{itemize}

\section{Workshop}

\LaTeX heeft al een paar standaard-layouts, maar vrijwilligers hebben er nog meer bijgemaakt, van universiteiten die een bepaalde layout-norm definiëren tot psychologen die weten hoe je CV het meeste impact maakt. Deze zelfgemaakte layouts heten templates.
In Share\LaTeX is het heel makkelijk om zo'n template te gebruiken, zonder dat je zelf aan de layout hoeft te denken: dat hebben designers al allemaal voor je bedacht. 

 We gaan eerst kijken naar een blanco project: 
\begin{itemize}
\item Klik op New Project\textgreater Blank project
\end{itemize}


Je bent nu in de ShareLaTeX Integrated Development Environment. Volg de tour om te zien wat alle knopjes doen, dan kunnen we daarna verder met het aanpassen van de layout.
\begin{itemize}
\item Klik eerst op Recompile, dit is hoe je document er nu uit ziet.
\item Het ziet er nu heel simpel uit, dus we gaan wat dingen aanpassen. 
\begin{description}
\item[documentclass.] Hier kan je de grootte van de tekst, grootte van het papier en het lettertype aanpassen. Jullie mogen alle woorden aanpassen, als er maar iets komt te staan wat \LaTeX begrijpt. Je kan altijd op ctrl+z drukken om je laatste wijzigingen ongedaan te maken.
\end{description}
\end{itemize}

\section{Plaatjes}
Wanneer je een plaatje toe wilt voegen moet je eerst bovenaan je document, onder de andere usepackage commando's, het volgende neerzetten:
\begin{verbbox}
\usepackage{graphicx}
\end{verbbox}
\theverbbox

Op shareLatex kan je plaatjes uploaden door linksboven op het menu op de knop 'upload' te klikken. 

Wanneer je vervolgens in je tekst dit plaatje wil toevoegen gebruik je het volgende commando: \\
\begin{verbbox}
\includegraphics[variabele1 = waarde1... variabelen = waarden]{naam}
\end{verbbox}
\theverbbox

Waar alles tussen de []-haakjes optioneel is, en de naam de naam van het plaatje \emph{zonder} extensie. Als je plaatje te groot is, kan je als een optionele variabele width of height toevoegen. Width kan meerdere types aan: cm, pt, en px, bijvoorbeeld. Er zijn nog meer variabelen die je kan aanpassen, maar die weet google wel voor je.


Waar moet je dit commando neerzetten? Dat ligt eraan waar je het plaatje wilt hebben (duh). Als je hem boven de titel wilt, moet je hem tussen \verb| \begin{document}| en \verb|\maketitle| zetten. Als je jouw kunstwerk ergens anders wilt hebben, kan je hem bijvoorbeeld in section Introduction neerzetten. 



En vervolgens een figure aan te maken waar je plaatje in staat:\\
\begin{verbbox}
  \begin{figure}[h]
    \centering
    \includegraphics[width=0.8\textwidth]{name}
    \caption{Awesome Image}
    \label{fig:awesome_image}
\end{figure}
\end{verbbox}
\theverbbox

De h achter begin figure staat voor 'here', hij plaatst hem dan ongeveer op die plek. Je kan de h ook vervangen door de t (voor de top van de pagina), de b (voor onderaan de pagina, of de p (voor op een aparte pagina). 

Met het label kan je later makkelijk verwijzen naar het plaatje, dit zien we terug in Sectie~\ref{sec:verwijzingen}. 

\section{Bulletpoints}
Nu we het toch over bulletpoints hebben: hoe maak je die in \LaTeX? Zo:

\begin{verbbox}
\begin{itemize}
\item item1 ... 
\item itemn
\end{itemize}
\end{verbbox}
\theverbbox

Wil je nummertjes? Gebruik dan enumerate:

\begin{verbbox}
\begin{enumerate}
\item item1 ... 
\item itemn
\end{enumerate}
\end{verbbox}
\theverbbox


\section{Tabellen}
Iets wat niet kan ontbreken aan een document zijn tabellen. Als je geen tabel in je document hebt, heb je iets fout gedaan. Het is even wennen om ze toe te voegen aan \LaTeX, maar als het eenmaal lukt, wil je ze ook overal aan toevoegen. Je maakt ze zo:

\begin{verbbox}
\begin{tabular}{ l || c r }
  1 & 2 & 3 \\
  \hline
  4 & 5 & 6 \\
  7 & 8 & 9 \\
\end{tabular}
\end{verbbox}
\theverbbox

Met \verb|tabular| geef je aan dat je een tabel wilt. Wat je daarna in de accolade's zet zijn de uitlijning en scheiding tussen cellen. \verb|l| betekent links uitgelijnd, \verb|c| centraal en \verb|r| rechts. Je kan een \verb|' '|, een enkele \verb+'|'+ of dubbele \verb+'||'+ scheiding tussen de kolommen doen. Voor een scheiding tussen de rijen wil je \verb|\hline| tussen de rijen zetten.\\
Verder maak je verschillende cellen door een \verb|&|-teken er tussen te zetten, en begin je een nieuwe rij met \verb|\\|.\\
Als je liever een generator gebruikt, kan je naar \href{http://www.tablesgenerator.com/}{tablesgenerator.com} gaan om dat voor je te laten doen.

\section{Wiskunde}
Ook wiskunde is iets wat LaTeX heel mooi kan maken. Er zijn heel veel verschillende packages aan toegewijd, maar met \verb|\usepackage{mathtools}| kan je de meeste dingen die je wilt hebben wel invoegen. Als je \verb|\[| en \verb|\]| om je formule heenzet, wordt hij wiskundig weergeven, en op een aparte regel. Als je het gewoon tussen woorden door wilt gebruiken, moet je er \verb|$...$| om heenzetten. Standaard-tekens die je op je toetsenbord kan vinden (zoals $+,-,<,>$), kan je gewoon gebruiken in de formule. Meer bijzondere tekens moet je invoeren als een commando, die je op internet moet vinden. Hieronder een tabel met een paar commando's die wel interessant zijn: 


\begin{tabular}{l | l}
\verb|\left(\frac{teller}{noemer}\right)| & \large{$\left(\frac{teller}{noemer}\right)$} \\ \hline
\verb|variabele_{index}^{exponent}| & \large{$variabele_{index}^{exponent}$} \\ \hline
\verb|\sqrt[optioneel]{\ldots}| & \large{$\sqrt[optioneel]{\ldots}$} \\ \hline
\verb|\sum_{n=0}^N p_n \log (p_n)| & \large{$\sum_{n=0}^N p_n \log (p_n)$} \\ \hline
\verb|F(x) = \int f(x) d x| & \large{$F(x) = \int f(x) d x$} \\ \hline
\verb|a \in Domein, a \to \neg b \lor c| & \large{$a \in Domein, a \to \neg b \lor c$} \\
\end{tabular}

"Maar," hoor ik je al vragen, "hoe maak ik dan een matrix?" Waarop ik antwoord: dat kan gewoon met een tabel, grapjas.

Er zijn nog allemaal dingen die je zelf kan aanpassen aan je CV, die ik niet heb opgeschreven. Voor meer inspiratie kan je de slides van het eerste deel van de workshop raadplegen. De slides staan \href{http://ctan.math.washington.edu/tex-archive/info/latex-course/}{hier}. Er staat ook een .tex bestandje van de slides, die je kan bekijken als je wilt.



\section{Verwijzingen}
\label{sec:verwijzingen}
Tijdens het schrijven wil je vaak verwijzen naar andere plekken in je document (i.e. een plaatje, een tabel of een sectie). Dit doe je door elk plaatje, sectie en tabel een label te geven. Voor een sectie doe je dit op deze manier: 
\begin{verbbox}
\section{Speciale sectie}
\label{sec:speciale-sectie}
\end{verbbox}
\theverbbox

Hier kan je vervolgens op de volgende manier naar verwijzen: 

\begin{verbbox}
Zie hiervoor Sectie \ref{sec:speciale-sectie}. 
\end{verbbox}
\theverbbox

Het is gewoonte om in latex verwijzingen die verwijzen naar een sectie een label te geven wat begint met sec: , plaatjes een label te geven wat begint met fig:, en tabellen een label te geven wat begint met tab: . 
\section{Conclusie}
Dat was het voor deze workshop. Je kan nog veel meer te leren door zelf dingen te proberen en vooral veel te Googlen! Als je nog dingen wilt weten kan je altijd een assistent vragen (als je nog bij de workshop bent) of mij contacteren. Ik hoop dat je veel profijt hebt aan het beheersen van \LaTeX.

\end{document}
